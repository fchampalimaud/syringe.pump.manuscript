\subsection*{Syringe Pump construction}
The presented syringe pump system is composed of a mechanical assembly, an electronic controller board, and an accompanying firmware/software stack that affords different levels of control over the system's behavior.

\subsubsection*{Mechanical construction}
The mechanical assembly of the syringe pump was designed to be easily produced and assembled without any particular expertise. All parts necessary for the build are included in the \TODO{BILL OF MATERIALS}. Most parts are readily available off the shelf. Additionally, custom structure parts can be in-house laser-cut and 3D printed, if such resources are available, or sourced from third-party manufacturers, using the provided designs. In addition to the provided bill of materials \TODO{ref here}, we also documented the assembly protocol with step-by-step instructions, including photographs, that should be followed to guarantee the best possible assembly of the structure.

In the provided designs, we opted to include a NEMA 17 Stepper motor (HT17-275 from Applied Motion) due to its high precision and torque rating. Specifically, the combination of the motor (1.8 $\deg$/step) together with a 0.8 mm pitch length driving rod leads to a theoretical linear resolution of 4 $\mu m$/step, which can be further increased by using step multipliers of up to 1/16.
Additionally, to characterize the system, we chose two glass syringes models with volumes routinely used for animal experiments (5 and 10 $mL$ \TODO{ref hamilton...}). Taking their cross-section diameter into consideration, the final single-full-step theoretical resolution will be 0.33 and 0.66 $\mu L / step$ for the 5 and 10 $mL$ syringe models, respectively.

It should be noted, however, that the present system should be able to accommodate a large range of other options (\textit{e.g.:} motors or syringes), without further modifications to the provided designs.

\subsubsection*{Electronics}

The syringe pump is controlled by a custom-made printed circuit board which is comprised of three main blocks: a microprocessor, an interface logic circuitry and, a motor driver, \ref{fig:PumpDrawing}.

The microprocessor block consists of an ATxmega microprocessor that implements the Harp \TODO{(harp repo)} protocol, a USB to serial UART interface and a stereo phone jack for synchronization between Harp devices. The Harp protocol allows the syringe pump to be controlled externally through the USB interface with any software program using the Harp API (\textit{e.g.:} Bonsai or Python).

The interface circuitry block consists of logic buffers that provide a direct low-level interface with the microstepping driver (\TODO{ADD REF}) to the user (bypassing the microcontroller block). Moreover, it provides access to the I/O breakout that can be used to trigger protocol delivery with other external, TTL-compatible, devices, supporting input voltages up to 5V. The voltage of the digital outputs can be configured in the board through a jumper to select between 3.3V and 5V voltage output logic.

The modularity of the board affords users the option to assemble a lower-cost and simplified version, without the microcontroller block, for applications wherein only low-level control is required. This significantly cheaper version is assembled using the same PCB schematic with minimal changes to the board components and can be found in an alternative bill of materials provided.

Despite having been designed for the herein described configuration, the design of the control system (including the Harp API) is compatible with other types of bipolar stepper motors (assuming an output drive capacity of up to 12V and ±1.7A) opening the doors to a wide array of applications that use these devices.

\subsubsection*{Syringe pump user control}

We designed the syringe pump system to allow three distinct levels of operation control that afford experimenters control over the pump according to their experimental needs. In addition to \ref*{fig:PumpControl}, we offer a brief description of the three available options:
\begin{itemize}

\item{Low-level control} - This option relies on directly controlling the I/O interface of the microstepping driver. This is achieved by using an external source (\textit{e.g.:} Arduino microcontroller) to generate the necessary input logic. Eight inputs lines are exposed: GND (ground), EN (enables the driver when low), MS1-3 (determines the step-size of the motor, 1 to 1/16 times a full step), DIR (determines the direction of rotation) and STEP (triggers microstepping). It should be noted that this control logic is common and documentation is extensively available \TODO{(REF)}.
Additionally, the PCB exposes two outputs that correspond to "end-of-travel" switches, that can be used to implement safety operation stops.
This control option does not require the full board to be assembled, as it does not rely on the microcontroller block for control and, while it affords the user the most flexibility, it requires an external source of logic control that might be cumbersome to set up and unnecessary for the vast majority of applications. Thus, the next two options abstract some of the low-level control away from the user at the cost of some flexibility.

\item{Trigger-based Control} - This option allows the user to configure a \textit{“protocol”} that is triggered with the detection of a TTL rising edge. This control logic is extensively used across many platforms (\textit{e.g.} Med associates, OpenEphys, Arduino, Harp) making the system easily integrable with existing experimental rigs.
The protocol can be configured via a custom-designed graphic user interface \ref{fig:PumpControl} that leverages the Harp protocol to configure the microcontroller. Available settings currently include an option to run in “Step Mode” (user sets number, size, and period of the steps per protocol) or “Volume Mode” (user sets flow rate, volume and, calibration settings). The behavior of the I/O pins can also be configured via this GUI to provide users with hardware events that can be used to interface with third-party hardware (e.g. electrophysiology, see figure \TODO{ADD Ephys reference})
It should be noted that, contrary to the previous mode, “end-of-travel” events (\textit{i.e.,} when the limit switches are triggered) are automatically handled by the controller, to prevent accidental mechanical damage to the system.

\item{Software Control} - The third option allows the user to control the syringe pump system using software alone. This option can be seen as an extension of the previous, second, option but with the protocol being triggered by a software event. This option is currently powered by Bonsai. Bonsai is a visual programming language that specializes in asynchronous data acquisition and has seen a steady growth of adoption from the neuroscience community. Users can change an extensive array of settings (identical to the “triggered control” version) and trigger a protocol on any Bonsai event. As an example, the user can periodically deliver a constant amount of reward \ref{fig:PumpControl}, dynamically change the amount of liquid delivered on every trial, contingent on some behavior/physiological event, or trigger the delivery of water aligned to when the animal enters a given area of an arena.
Finally, using the Harp protocol allows the user access to hardware timestamped events (e.g. motor stepping) making event alignment possible and effortless.

\end{itemize}

\subsection*{Calibration}
\subsubsection*{Large volume calibration}
As mentioned before, the pump is compatible with a wide variety of syringes. However, it is advisable to calibrate the full system to account for potentially significant differences across parts. Small volumes of delivered liquid are technically challenging to measure, as a result, calibrating liquid delivery systems is usually performed by repeating the same protocol a large number of times. This strategy rests on the assumption that single steps are relatively reproducible and the mean across several dozens to hundreds of repetitions is thus representative. We calibrated the system by varying the number of steps per protocol, which was then repeated 200 times, waiting 250 ms between the end of each protocol and the start of the next. We then weighted the amount of collected liquid to yield the total collected volume (we assumed a water density of 1g/mL at room temperature). Single protocol volume was thus given by \ref{eq:LargeVolumeCalibration}:

\begin{equation} \label{eq:LargeVolumeCalibration}
    V_{single}(mL) = \frac{Weight (g)}{1 (g/mL)} * \frac{1}{200}
    \end{equation}

To assess reproducibility across runs, we repeated this protocol several times (20). Depending on the experiment, we also adjusted the range of calibration values. Critically, all tested volumes lay on top of a linear calibration regime, and outcomes were consistent with the predicted theoretical volumes \ref{eq:LargeVolumeCalibration}.

We followed protocol identical to the syringe pump system to calibrate the solenoid valves (Model LHDA1231215H, Lee Co.). Briefly, we calibrated the valves by setting the water reservoir (a 20mL plastic syringe) at a height identical to what we routinely use in our laboratory for behavior experiments (~30cm). We then systematically varied the opening time (between 10 and 150 ms) to yield different delivery volumes. Both the number and time between trials were identical to the ones used in the pumps. Hardware control logic was implemented in Arduino.

\subsubsection*{Single bolus calibration}

While technically amenable, measuring the outcome of several hundred protocols might obscure variability across single trial events. We thus developed a simple, computer-vision-based, method to measure volumes delivered in single protocols. Briefly, the end of the syringe was fitted with a plastic adapter with a small glass capillary (Model 211713, Vitrex) glued to the end (\ref{fig:PumpDrawing}). A computer vision camera (Model FL3-U3-13S2C-CS, FLIR) was used to image the capillary.
In order to increase the contrast between liquid and background, we diluted a small amount of red colorant (Erythrosin B, Sigma-Aldrich, 198269) combined with LED illumination, resulting in the image shown in \ref{fig:PumpProtocol}.

The video acquisition and processing routine were implemented in Bonsai \TODO{GITHUB}. Briefly, to measure the amount of displaced liquid inside the capillary, we defined a region of interest, excluding optical artefacts resulting from glass diffraction, and binarized the image thresholding the pixel values color space. We then used this binarized image to segment the area corresponding to the displaced liquid volume. We choose the horizontal axis of the region as our reported metric. Data were acquired at 10 frames-per-second and synchronization was achieved by sending a short TTL pulse to the camera each time a protocol was started. Syringe pump control logic was implemented using the low-level control mode of the system and an Arduino microcontroller.

Unless otherwise stated, the inter-pulse-interval was set at 4 ms. Before each run, up to 5 small pulses were applied to ensure that every protocol started from comparable conditions. These pulses were discarded from all analyses. The interval between each protocol was drawn from a uniform distribution (10 to 20 s). Due to the size of the capillary, some volume combinations required cleaning the tube between runs. We cleaned it by flushing ethanol followed by air-drying it.

For the dynamic flow rate experiment shown in \ref{fig:FlowRateControl}, the inter-pulse-interval was systematically changed in order to control the output flow rate. We estimate flow rate by linearly fitting, on each trial, the displaced area to time (between protocol onset and the stabilized epoch).

\subsection*{Experimental validation}

\subsubsection*{Animal behavior experiments}
Two Long Evans, 5 month-old, male rats were trained on a variant of a Two-armed bandits task \TODO{REF} to assess reward preference. Rats were placed in an experimental box with access to four nose ports, two for initiation, and two for reward delivery. Reward volumes were drawn from a pre-defined set of calibrated values (3.46, 6.92, 13.84, 27.68, and 55.36 \TODO{uL}) and delivered using the presented syringe pump system. A light on top of the initiation ports signalled the availability of a trial. After initiating the trial, animals were required to fixate at the centre nose port for a short period of 100 milliseconds, after which they would be allowed to collect a reward at one of the two side-reward ports. Importantly, the amount of water delivered at any port was constant within a given session block, but always different across the two ports. After reaching a preference criteria of >80\% towards the highest reward nose-port over the last 15 trials, a block transition would be triggered, the trial initiation switched to the opposite nose port (\textit{e.g.:} odd and even numbered blocks would be initiated at the north and south nose ports, respectively). At block transitions, we required that the rewards on each nose-port could not be equal, but otherwise allowed for any other combination. In particular, the highest rewarded nose-port was not enforced to change between blocks. As a result, block switch did not imply a change of the highest-rewarded nose-port.


\subsubsection*{Compatibility with electrophysiology recordings}

Previous studies report induced electrical artefacts from some syringe pump systems \TODO{ref}. We thus tested the compatability of our system with electrophysiological techniques by recording these signals while triggering the microstepping driver. 
Briefly, we followed the protocol for acute recordings described in \cite{Cruz2022}. Mice (8-10 weeks old) were anesthetized with isoflurane (1-2 \% at 0.8L/min) and a small straight metalic headpost was securely cemented to the skull. Animals were given a dose of Carprofen on the day of the surgery and were allowed to recover for a minimum of 5 days. Prior to the recording session, a small circular craniotomy (1.5mm of diameter) was opened above the dorsal striatum where recordings were performed. During data acquisition, animals were head-restrained by a custom-made apparatus and allowed to run on top of a rotatable cylinder. A silicon probe (ASSY 77-H2, Cambridge NeuroTech) was slowly lowered (2-10um/s) into the dorsal striatum (~3mm from brain surface) after which we waited a minimum of 30min before recordings began. Headstage (Intan) data was digitized and 30kHz and acquired using Open Ephys Acquisition board and Bonsai. In order to align electrophysiology data with pump events, we used the aforementioned “Triggered Control” mode and configured one of the controller’s outputs to report STEP events via TTL, which was then acquired with the same Open Ephys acquisition board. The syringe pump was positioned ~20cm away from the animal without any shielding material in-between. Protocols were manually triggered by the experimenter.
Raw electrophysiological data was minimally processed by applying a digital bandpass Butterworth filter (0.3 to 8kHz).


